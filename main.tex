% Template:     Informe/Reporte LaTeX
% Documento:    Archivo principal
% Versión:      3.2.0 (24/04/2017)
% Codificación: UTF-8
%
% Autor: Pablo Pizarro R.
%        Facultad de Ciencias Físicas y Matemáticas.
%        Universidad de Chile.
%        pablo.pizarro@ing.uchile.cl, ppizarror.com
%
% Sitio web del proyecto: [http://ppizarror.com/Template-Informe/]
% Licencia: MIT           [https://opensource.org/licenses/MIT]

% CREACIÓN DEL DOCUMENTO, FUENTE E IDIOMA
\documentclass[letterpaper,11pt]{article} % Articulo tamaño carta, fuente 11
\usepackage[utf8]{inputenc}               % Codificación UTF-8
\usepackage[T1]{fontenc}                  % Soporta caracteres acentuados
\usepackage{lmodern}                      % Tipografía moderna
\usepackage[spanish]{babel}               % Idioma del documento en español
\def\templateversion{3.2.0}               % Versión del template
                                            
% INFORMACIÓN DEL DOCUMENTO
\def\nombredelinforme {Laboratorio N°3}
\def\temaatratar {Intercambiadores de Calor}
\def\fecharealizacion {\today}
\def\fechaentrega {\today}

\def\autordeldocumento {Grupo 10}
\def\nombredelcurso {Transferencia de calor}
\def\codigodelcurso {ME-4302-1}

\def\nombreuniversidad {Universidad de Chile}
\def\nombrefacultad {Facultad de Ciencias Físicas y Matemáticas}
\def\departamentouniversidad {Departamento de Ingeniería Mecánica}
\def\imagendeldepartamento {images/departamentos/dimec}
\def\imagendeldepartamentoescl {0.2}
\def\localizacionuniversidad {Santiago, Chile}

% INTEGRANTES, PROFESORES Y FECHAS
\newcommand{\tablaintegrantes}{
\begin{minipage}{1.0\textwidth}
\begin{flushright}
\begin{tabular}{ll}
	Integrantes:
		& \begin{tabular}[t]{@{}l@{}}
			Leonardo Garrido \\
			Cristopher Gómez\\   
			Rodrigo Moraga
		\end{tabular} \\
	Profesores:
		& \begin{tabular}[t]{@{}l@{}}
			Álvaro Valencia \\
			
		\end{tabular} \\
	Auxiliares:
		& \begin{tabular}[t]{@{}l@{}}
			Ian Wolde \\
			
		\end{tabular}\\
	Ayudantes de laboratorio:
		& \begin{tabular}[t]{@{}l@{}}
			Erick Kracht \\
			Ricardo Jadue
		\end{tabular}\\
	%\multicolumn{2}{l}{Ayudante del laboratorio: Ayudante} \\
	& \\
	\multicolumn{2}{l}{Fecha de realización: 20 de Noviembre de 2018} \\
	\multicolumn{2}{l}{Fecha de entrega: 30 de Noviembre de 2018} \\
	\multicolumn{2}{l}{\localizacionuniversidad}
\end{tabular}
\end{flushright}
\end{minipage}}

% CONFIGURACIONES
\input{lib/config}

% IMPORTACIÓN DE LIBRERÍAS
\input{lib/imports}

% IMPORTACIÓN DE FUNCIONES
\input{lib/functions}

% IMPORTACIÓN DE AMBIENTES Y ESTILOS
\input{lib/styles}

% CONFIGURACIÓN INICIAL DEL DOCUMENTO
\input{lib/initconf}

% INICIO DE LAS PÁGINAS
\begin{document}
	
% PORTADA
\input{lib/portrait}

% CONFIGURACIÓN DE PÁGINA Y ENCABEZADOS
\input{lib/pageconf}

% RESUMEN O ABSTRACT
\input{abstract} % Se incluye un ejemplo de resumen, se puede borrar

% TABLA DE CONTENIDOS - ÍNDICE
\input{lib/index}

% CONFIGURACIONES FINALES - INICIO DE LAS SECCIONES
\input{lib/finalconf}

% ========================== INICIO DEL DOCUMENTO ==========================

\section{Introducción}

Los intercambiadores de calor son utilizados para transmitir calor desde un fluido a otro. El intercambio de energía suele realizarse a través de una pared termoconductora, la que impide que los fluidos se mezclen y que debido a esto se contaminen.
\newp 
Las aplicaciones para los intercambiadores de calor son diversas, como aparatos de refrigeración, radiadores de auto, sistemas de aire, calefón, calderas, etc.

\newp
La transmisión de calor dentro de un intercambiador de contacto indirecto se produce desde el medio caliente al medio frío.
El fluido a mayor temperatura calienta la pared termoconductora, provocando que se el primero se enfríe y que la pared se caliente, luego, el calor transmitido por el fluido caliente a la pared es cedido al fluido a menor temperatura, provocando que su temperatura aumente. 
Por lo anterior, en un intercambiador de calor, la temperatura de salida del fluido caliente siempre es menor a la de entrada, mientras que la del fluido frío es mayor.





\newpage
\section{Objetivos}
\subsection{Objetivo general}
Estudiar la transferencia de calor en intercambiadores de calor de placas planas cuando se tiene corriente paralela y corriente inversa.

\subsection{Objetivos específicos}
\begin{itemize}
    \item Aprender el funcionamiento del equipo WL-110.02.
    \item Controlar los caudales de los fluidos del intercambiador de calor.
    \item Medir temperatura en los puntos de entrada y salida del intercambiador.
    \item Calcular el calor intercambiado a partir de las mediciones de temperatura y caudal.
    \item Calcular el coeficiente medio de transmisión térmica a partir del calor intercambiado y los parámetros geométricos del intercambiador.
\end{itemize}
		
\newpage
\section{Antecedentes}
Los intercambiadores de calor se utilizan para transmitir calor desde un fluido a otro sin la necesidad de mezclarlos. El calor se transmite a través de una pared termoconductora que separa ambos fluidos.
Se tiene entonces un proceso de tres partes: transmisión de calor por parte del fluido caliente a la pared, conducción de calor a través de la pared y transmisión de calor por parte de la pared al fluido frío.
\subsection{Conducción de calor}
La ecuación que describe la transmisión de calor de un medio a la pared (o de la pared a n medio) corresponde a:

\begin{equation}
    Q = \alpha \cdot A \cdot \Delta T \cdot t
\end{equation}
 Donde, 
 \begin{itemize}
     \item $Q$ es el calor transferido.
     \item $\alpha$ es el coeficiente de transferencia de calor del material.
     \item $A$ es el área transversal por el cual ocurre el intercambio de calor.
     \item $\Delta T$ Es la diferencia de temperatura entre el medio y la pared.
     \item $t$ es el tiempo que transcurrido.
 \end{itemize} 
 
 Derivando la expresión (3.1) respecto del tiempo, se obtiene la taza de intercambio de calor :
 
 \begin{equation}
     \dot Q = \alpha \cdot A \cdot \Delta T
 \end{equation}

 Donde $\dot Q$ es el flujo calorífico medido en unidades de potencia.
 
La relación que describe la conducción de calor a través de la pared es la siguiente:
\begin{equation}
    \Dot{Q} = \frac{\lambda A \Delta T_w}{s}
\end{equation}
\begin{itemize}
    \item $\lambda$ coeficiente de conductibilidad térmica del material $[\frac{W}{m\cdot K}]$.
    \item $A$ el área transversal a la dirección del flujo de calor $[m^2]$.
    \item $\Delta T_w$ corresponde a la diferencia de temperatura entre los bordes de la pared $[K]$.
    \item $s$ el espesor de la pared $[m]$.
\end{itemize}

Si se asume estado estacionario el flujo de calor del medio caliente a la pared, el flujo a través de la pared y el flujo de calor de la pared al medio frío son equivalentes.  Por lo que se puede definir un coeficiente medio de transmisión térmica $k_m$:
\begin{equation}
    k_m = \frac{1}{\frac{1}{\alpha_1}+\frac{s}{\lambda}+\frac{1}{\alpha_2}}
\end{equation}
Por lo que el flujo calórico viene dado por la ecuación:
\begin{equation}
    \Dot{Q} = k_m A \Delta T
\end{equation}

Con $\Delta T = T_1 - T_2$ es decir la diferencia de temperatura entre los medios.

Debido a que los fluidos del intercambiador no tienen una temperatura constante a lo largo de este, se utiliza la temperatura logarítmica media para calcular el calor transferido:
\begin{equation}
    \Dot{Q} = k_m A_m \Delta T_{ln}
\end{equation}
\begin{equation}
    \Delta T_{ln} = \frac{\Delta T_{max}-\Delta T_{min}}{\ln{ \frac{\Delta T_{max}}{\Delta T_{min}}}}
\end{equation}
\begin{equation}
    A_m = \frac{A_1-A_2}{\ln{\frac{A_1}{A_2}}}
\end{equation}

\subsection{Calculo de flujo de calor}
Otra relación para calcular el flujo de calor intercambiado utiliza la capacidad térmica específica de los fluidos $c_p$, el caudal másico $\dot{m}$ y la diferencia de temperatura entre la salida y la entrada $T_{A}-T_{E}$
\begin{equation}
\dot{Q} = \dot{m}c_p (T_{A}-T_{E})
\end{equation}
Esta relación puede ser utilizada para ambos fluidos. Si no existe pérdida, los valores obtenidos son inversos aditivos. Si existe pérdida se puede utilizar el valor medio entre ambos flujos de calor:
\begin{equation}
\dot{Q}_{wm} = \frac{(-\dot{Q}_{w1}) + \dot{Q}_{w2}}{2}
\end{equation} 
\subsection{Coeficiente medio de transmisión térmica}
Con el flujo de calor medio calculado a partir de mediciones de temperatura y caudal, se puede calcular el coeficiente medio de transmisión térmica $k_m$ que caracteriza el intercambiador de calor:
\begin{equation}
k_m = \frac{\dot{Q}_{wm}}{A_m \Delta T_{ln}}
\end{equation}
\subsection{Grado de efectividad del intercambiador}
Adicionalmente es posible caracterizar al intercambiador según el grado de efectividad $\eta_{cal}$ para calentamiento y $\eta_{enf}$ para enfriamiento:
\begin{equation}
\eta_{cal} = \frac{\dot{Q}_{w1}}{\dot{Q}_{w2}}
\end{equation} 
\begin{equation}
\eta_{enf} = \frac{\dot{Q}_{w2}}{\dot{Q}_{w1}}
\end{equation}



\newpage

\section{Metodología}
\subsection{Equipo a utilizar}
El intercambiador de calor que se utiliza corresponde a un intercambiador de placas (Figuras \ref{fig:elev-inter} y \ref{fig:planta-inter}). Existe una fuente controlable de agua caliente y fría que adicionalmente entrega mediciones de temperatura en la entrada y salida del intercambiador. 
\insertimage[\label{fig:elev-inter}]{elev.png}{width=7cm}{Intercambiador de placas. Vista en elevación.}
\insertimage[\label{fig:planta-inter}]{planta.png}{width=7cm}{Intercambiador de placas. Vista en planta.}
\subsection{Experiencias}
Se realizan dos experiencias. En la primera experiencia se conecta la fuente de agua al intercambiador de tal forma que tanto el flujo caliente y flujo frío se encuentran en corriente paralela. En la segunda experiencia se invierte la salida con la entrada del fluido frío. De esta forma se tiene un intercambiador de calor con corriente inversa.

Para cada experiencia se varía el caudal del flujo caliente y del flujo frío para tener 6 pares de mediciones distintas de caudales. Para cada par de datos de caudal se mide la temperatura en la salida y entrada para el flujo frío y para el flujo caliente.

Con los datos de caudal, temperatura, propiedades del agua y parámetros geométricos del intercambiador de calor, se realizan cálculos para el flujo de calor intercambiado y para el coeficiente medio de transmisión térmica del intercambiador. 

\newpage
\section{Memoria de Cálculo}
\subsection{Caudal másico}
\begin{equation}
\dot{m} = \rho \dot{V}
\end{equation}
\begin{itemize}
	\item $\dot{m}$ caudal másico $[\frac{kg}{s}]$
	\item $\rho$ densidad del fluido $[\frac{kg}{m^3}]$
	\item $\dot{V}$ caudal volumétrico $[\frac{m^3}{s}]$
\end{itemize}
\subsection{Temperatura media logarítmica}
\begin{equation}
\Delta T_{ln} = \frac{\Delta T_{max}-\Delta T_{min}}{\ln{ \frac{\Delta T_{max}}{\Delta T_{min}}}}
\end{equation}
\begin{itemize}
	\item $\Delta T_{ln}$ temperatura media logarítmica $[K]$
	\item $\Delta T_{max}$ máximo entre las diferencias de temperatura entre salida y entrada para flujo caliente y frío $[K]$
	\item $\Delta T_{min}$ mínimo entre las diferencias de temperatura entre salida y entrada para flujo caliente y frío $[K]$
\end{itemize}
\subsection{Flujo de calor intercambiado}
\begin{equation}
\dot{Q} = \dot{m}c_p (T_{A}-T_{E})
\end{equation}
\begin{itemize}
	\item $\dot{Q}$ flujo de calor intercambiado $[W]$
	\item $\dot{m}$ caudal másico $[\frac{kg}{s}]$
	\item $c_p$ capacidad térmica específica del fluido $[\frac{J}{kg\cdot K}]$
	\item $T_{A}$ temperatura a la salida $[K]$
	\item $T_{E}$ temperatura a la entrada $[K]$ 
\end{itemize}
\subsection{Flujo de calor medio}
\begin{equation}
\dot{Q}_{wm} = \frac{(-\dot{Q}_{w1}) + \dot{Q}_{w2}}{2}
\end{equation} 
\begin{itemize}
	\item $\dot{Q}_{wm}$ flujo de calor medio $[W]$
	\item $\dot{Q}_{w1}$ flujo de calor en el medio caliente $[W]$
	\item $\dot{Q}_{w2}$ flujo de calor en el medio frío $[W]$
\end{itemize}
\subsection{Grado de efectividad del intercambiador}
\begin{equation}
\eta_{cal} = \frac{\dot{Q}_{w1}}{\dot{Q}_{w2}}
\end{equation} 
\begin{equation}
\eta_{enf} = \frac{\dot{Q}_{w2}}{\dot{Q}_{w1}}
\end{equation}
\begin{itemize}
	\item $\eta_{cal}$ grado de efectividad para calentamiento
	\item $\eta_{enf}$ grado de efectividad para enfriamiento
	\item $\dot{Q}_{w1}$ flujo de calor en el medio caliente $[W]$
	\item $\dot{Q}_{w2}$ flujo de calor en el medio frío $[W]$
\end{itemize}
\subsection{Coeficiente medio de transmisión térmica}
\begin{equation}
k_m = \frac{\dot{Q}_{wm}}{A_m \Delta T_{ln}}
\end{equation}
\begin{itemize}
	\item $k_m$ coeficiente medio de transmisión térmica $[\frac{W}{m^2\cdot K}]$
	\item $\dot{Q}_{wm}$ flujo de calor medio $[W]$
	\item $A_m$ superficie de intercambio de calor media logarítmica $[m^2]$
	\item $\Delta T_{ln}$ temperatura media logarítmica entre los flujos $[K]$
	
\end{itemize}
\newpage
\section{Resultados}
Los resultados para el experimento de el intercambiador con flujo paralelo se presentan en la siguiente tabla, cabe destacar que se utiliza la misma nomenclatura usada por la guía de laboratorio y por el equipo experimental:
\newp
\begin{table}[htbp]
    \centering
    
    \caption{Datos tabulados para la experiencia de flujo paralelo}
    \begin{tabular}{c|c|c|c|c|c}
    V1 $[\frac{l}{min}]$ & V2 $[\frac{l}{min}]$ &T1[$^{°}$C]& T3[$^{°}$C]&T4[$^{°}$C]&T6[$^{°}$C]\\
        \hline
        1.305 & 1.38 & 56.55 &42.9 &23.05 &36.6\\
        1.32 & 3.09 & 55.35 &38.45 &23.25 &31\\
        2.005 & 1.33 & 57.4 &46.25 &22.8 &39.4\\
        2.02 & 2.975 & 56.25 &42.1 &23.05 &33.25\\
    \end{tabular}
    \label{tab:paralelo}
\end{table}
\newp
Para el caso del flujo anti-paralelo se obtuvo lo siguiente:
\begin{table}[htbp]
    \centering
    \caption{Datos tabulados para la experiencia de flujo antiparalelo}
    \begin{tabular}{c|c|c|c|c|c}
    
        V1 $[\frac{l}{min}]$ & V2 $[\frac{l}{min}]$ &T1[$^{°}$C]& T3[$^{°}$C]&T4[$^{°}$C]&T6[$^{°}$C]\\
        \hline
        1.325 & 1.35 & 56.1 &40.25 &22.4 &38.3\\
        1.32 & 3.035 & 55.25 &33.6 &22.3 &31.2\\
        1.955 & 1.305 & 57.95 &45.15 &22.9 &42.25\\
        1.95 & 3.025 & 56.7 &40.6 &22.65 &33.9\\
    \end{tabular}
    \label{tab:antiparalelo}
\end{table}


\newpage
\section{Análisis de Resultados}



\newpage
\section{Conclusiones}.




% FIN DEL DOCUMENTO
\end{document}